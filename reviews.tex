\documentclass{article}
\usepackage[margin=1in]{geometry}
\usepackage{fancyhdr}
\pagestyle{plain}

\begin{document}
	
\section*{ChIP--Seq identification of weakly conserved heart enhancers \cite{blow-naturegen-2010}}
	Few enhancers regulating heart development have been identified. The collective interplay of these
	elements therefore yield accurate and sound heart development. Methods such as sequence conservation have 
	been used to identify conserved enhancers, however such methods often fail to identify heart enhancers.
	This study uses ChIP--Seq and E11.5 mouse embryos to quantify over 3,000 prospective heart enhancers.
	In earlier studies, it was found that less than 2\% of empirically validated enhancers regulated heart development. 
	This small percentage is in contrast to 16\%, 14\%, and 5\% of enhancers which regulate forebrain, midbrain, and
	limbs, respectively.
	
	The transcriptional co--activator, p300, is ubiquitous in developing embryos and has been shown to bind to enhancers.
	Thus, using this factor could help identify potentially viable heart enhancers. This study performed p300 ChIP--Seq on
	270 mouse embryos, identified 3,597 potential enhancer peaks which do not overlap with known promoters.
	Interestingly, many (84\%) of these potentially enhancer peaks do not overlap peaks associated with the aforementioned
	brain segments. Evolutionary conservation of these potential heart enhancers revealed many (65\%) to be conserved amongst placental
	mammals. Only 6\% of heart enhancers overlapped genomic regions subject to high levels of constraint. This is in
	contrast to 44\%, 39\%, and 30\% of genomic regions with overlapping forebrain, midbrain, and limb enhancers.
	Thus, these findings reveal significant differences in evolutionary sequence constraint solely based on 
	p300 binding. As a result, this may be indicative as to why there is poor performance when using sequence conservation
	to identify potential heart enhancers.
	Transgenic assays were created for 130 candidate heart enhancers in E11.5 mouse embryos. Of these 97 sequences exhibited
	reproducible expression from which 81 were active in the embryo heart. Many empirically--validated enhancers were located close to
	genes involved in heart development and its function. Within 100kb of genes annotated to be involved in ``heart development'', there
	was more than a three--fold enrichment of p300 peaks. To serve as a baseline, there were no ``heart development" enrichments referencing 
	forebrain p300 peaks. 

% Reference bibtex entries
\bibliography{refs}
\bibliographystyle{ieeetr}

\end{document}