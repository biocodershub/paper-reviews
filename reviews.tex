\documentclass{article}
\usepackage[margin=1in]{geometry}
\usepackage{fancyhdr}
\pagestyle{plain}

\begin{document}
	
\section*{ChIP--Seq identification of weakly conserved heart enhancers \cite{blow-naturegen-2010}}
	Few enhancers regulating heart development have been identified. The collective interplay of these
	elements therefore yield accurate and sound heart development. Methods such as sequence conservation have 
	been used to identify conserved enhancers, however such methods often fail to identify heart enhancers.
	This study uses ChIP--Seq and E11.5 mouse embryos to quantify over 3,000 prospective heart enhancers.
	In earlier studies, it was found that less than 2\% of empirically validated enhancers regulated heart development. 
	This small percentage is in contrast to 16\%, 14\%, and 5\% of enhancers which regulate forebrain, midbrain, and
	limbs, respectively.
	
	The transcriptional co--activator, p300, is ubiquitous in developing embryos and has been shown to bind to enhancers.
	Thus, using this factor could help identify potentially viable heart enhancers. This study performed p300 ChIP--Seq on
	270 mouse embryos, identified 3,597 potential enhancer peaks which do not overlap with known promoters.
	Interestingly, many (84\%) of these potentially enhancer peaks do not overlap peaks associated with the aforementioned
	brain segments. Evolutionary conservation of these potential heart enhancers revealed many (65\%) to be conserved amongst placental
	mammals. Only 6\% of heart enhancers overlapped genomic regions subject to high levels of constraint. This is in
	contrast to 44\%, 39\%, and 30\% of genomic regions with overlapping forebrain, midbrain, and limb enhancers.
	Thus, these findings reveal significant differences in evolutionary sequence constraint solely based on 
	p300 binding. As a result, this may be indicative as to why there is poor performance when using sequence conservation
	to identify potential heart enhancers.
	Transgenic assays were created for 130 candidate heart enhancers in E11.5 mouse embryos. Of these 97 sequences exhibited
	reproducible expression from which 81 were active in the embryo heart. Many empirically--validated enhancers were located close to
	genes involved in heart development and its function. Within 100kb of genes annotated to be involved in ``heart development'', there
	was more than a three--fold enrichment of p300 peaks. To serve as a baseline, there were no ``heart development" enrichments referencing 
	forebrain p300 peaks. 

\section*{Conserved expression without conserved regulatory sequence: the more things change, the more they stay the same \cite{weirauch-trendsgen-2010}}
	 It has been shown that much a mammalian constrained sequence is outside of exons whereby changes in the binding sites of
	 regulatory proteins within these regions can drive evolutionary processes. One could easily state that sequence conservation is
	 reflective of functional conservation. However, there are increasing studies which show that function can remain conserved even
	 though its \textit{cis}--regulatory element is otherwise. Regulatory elements from orthologous genes could be validated using
	 reporter assays. Indeed, many examples exist where such genes are conserved in their phenotypic functionality and starkly different in 
	 their regulatory signatures. A supporting study examined transgenic zebrafish which drove the \textit{RET} gene using human enhancers. Surprisingly, 
	 \textit{RET} expression be--it using human or zebrafish enhancers, were close to one another.
	 Conservation could also take the form of coregulation. For instance, if two orthologs are correlated across species, then their coregulation 
	 is also conserved.	Within a single genome, binding site arrangements vary significantly amongst genes. This therefore indicates that many ways exist to
	 facilitate conserved physical expression.
	 
	 There are two main ways in which a regulatory sequence could change whilst maintaining the same phenotype.
	 The first entails a change in a \textit{trans}--regulatory sequence as a result of changes in DNA sequence. 
	 Second, \textit{cis}--elements could rearrange their spatial positions, ultimately leading to
	 the same phenotype. The former process entails was well--studied in yeast and examines switching of transcription factors (TFs)
	 in \textit{C. albicans}; ancestor and \textit{S. cerevisiae}. The study shows that in \textit{C. albicans}, RPs are regulated by
	 \textit{Tbf1}, however after the \textit{C. albicans}--\textit{S. cerevisiae} split, RPs were regulated by \textit{Tbf1} and \textit{Rap1}.
	 In \textit{S. cerevisiae}, it was shown that RPs are rather regulated by \textit{Rap1}. Thus, changes in \textit{cis}--element architecture
	 and underlying DNA between these species were the basis of the RP switch. Along these lines, what could even happen is that TFs regulating ancestoral
	 processes may be switched--out for other TFs which regulate the same exact process. For instance, in \textit{C. albicans}, galactose metabolism (GALs) 
	 were regulated by \textit{GalX}. After the split, \textit{S. cerevisiae} regulated \textit{GalX} using \textit{Mig1} and \textit{Gal4} TFs.
	 The former way whereby changes in a regulatory element yield the same phenotype entails \textit{cis}--regulatory element ``turnover''.
	 In other words, regulatory elements are gained and lost over very large time--frames ($10^6$--$10^8$ years).
	 Somewhat similar to turnover is ``shuffling'' whereby elements relocate to another segment. Intuitively, it is evident
	 how shuffling and turnover yields the same phenotype. After all, enhancers operate in a strand--independent manner and contain
	 many binding sites. 
	 
	 All--in--all, the paper discusses several important concepts worthy of mentioning. First, why are regulatory elements prone to change 
	 all--while their resultant phenotype remains the same? Second, are there long--term benefits to altering regulatory mechanics of a functioning
	 and sound regulatory element? There may be ``organizational schemes'' associated with such continuous  adjustments; a way to foster neofunctionality.
	 On a different note, \textit{cis}--element redundancy may facilitate avoiding deleterious mutations.

% Reference bibtex entries
\bibliography{refs}
\bibliographystyle{ieeetr}

\end{document}